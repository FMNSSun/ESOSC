\title{ESOSC-2014-D0}


\documentclass[paper=a4, fontsize=11pt]{scrartcl}

\usepackage[protrusion=true,expansion=true]{microtype}	
\usepackage{amsmath,amsfonts,amsthm}
\usepackage[pdftex]{graphicx}	
\usepackage{url}
\usepackage[T1]{fontenc}
\usepackage{fourier}
\usepackage[english]{babel}
\usepackage{syntax}
\usepackage{sectsty}
\allsectionsfont{\centering \normalfont\scshape}
\usepackage{fancyhdr}
\pagestyle{fancyplain}
\fancyhead{\horrule{0.5pt}}											
\fancyfoot[L]{ESOSC-2014-1}											
\fancyfoot[C]{}											
\fancyfoot[R]{\thepage}									
\renewcommand{\headrulewidth}{0pt}			
\renewcommand{\footrulewidth}{0pt}
\setlength{\headheight}{13.6pt}

\numberwithin{equation}{section}		
\numberwithin{figure}{section}			
\numberwithin{table}{section}				

\setlength\parindent{0pt}

\newcommand{\horrule}[1]{\rule{\linewidth}{#1}}

\title{
	\usefont{OT1}{bch}{b}{n}
	\normalfont \normalsize \textsc{Esoteric Standard Committee} \\ [25pt]
	\horrule{0.5pt} \\[0.4cm]
	\huge The Esoteric Standard Committee \\
	\horrule{2pt} \\[0.5cm]
}

\author{
	\normalfont\normalsize
	ESOSC-2014-1-2\\[-3pt]\normalsize
	\textbf{DRAFT}\\
	\today
}
\date{}



\begin{document}
\maketitle

\begin{center}
This standard has not been approved yet.
\end{center}

\section{About ESOSC}

The Esoteric Standard Committee (usually referred to as ESOSC) "approves" esolang
related standards to strive for a higher quality of descriptions (specifications)
of esoteric programming languages and esolang related works. This is to ensure 
that descriptions of esoteric programming languages (or related works) are 
detailed and unambigious enough for others to implement and understand them 
correctly. \\


Approval doesn't mean the ESOSC recommends using a standard. In fact,
you probably shouldn't because they aren't intended for serious purposes. Like
esoteric programming languages themselves, they are more intended for the
purpose of entertainment. \\


Ask your psychiatrist whether esoteric programming languages (esolangs) are
right for you. 


\section{EUIN}

The ESOSC Unique Identification Number uniquely identifies (usually referred to
as EUIN) a standard with respect to specific versions of certain standards.  


\subsection{Format}

The format of an EUIN is as follows:  

\begin{figure}
\begin{verbatim}
<EUIN> ::= 
  `ESOSC' <hyph> <year> <hyph> <an> 
          <hyph> <version> [<hyph> <suffix>]
<hyph> ::= `-' 
<an> ::= <number>
<version> ::= <number>
<number> ::= <digit> | <digit> <number> 
<digit> ::= `0'|`1'|`2'|`3'|`4'|`5'|`6'|`7'|`8'|`9' 
<year> ::= <digit> <digit> <digit> <digit> 
<suffix> ::= `D' | `A' | `O'
\end{verbatim}
\end{figure}
	
	
'year' denotes the year when the first version of the standard was created
(but not necessarily approved in the same year). 'an' is a unique number 
allocated within the year the first version of the standard was created. The
'version' is a monotonically increasing number starting at 1 and denotes
the version of the standard. 'an' is refered to as the Assigned Number (AN). \\


To indicate the state of a document it may be suffixed with a suffix. 
D indicates that the document is in draft state, A indicates that the document
is an approved standard, O indicates that the document is obsolete. Use of
suffix is generally not recommended. \\


'year' together with 'an' uniquely identifies a standard whereas 'year', 'an' 
and 'version' combined uniquely identify a document (which is a specific version
of a standard).

\subsubsection{Assigned Number}

The Assigned Number is allocated from the Assigned Number Block (ANB) of the member
creating the first draft of the standard. 

\subsection{Assigned Number Block}

Each member has a number block from which they can allocate Assigned Numbers
for drafts. Each member receives a (continious) block of 128 numbers upon 
becoming a member. 

The ESOSC manages a list of Assigned Number Blocks. 


\section{Approval of standards }

For a standard to be approved at least three members of the ESOSC have to
approve it without having been bribed, forced, blackmailed or influenced by
drugs, alcohol or similar substances to do so. 


\section{Obsoletion}

Standards may be marked as obsolete when they are no longer
appropriate to use or have been completely replaced by another standard and can
The submitter of the standard can obsolete the standard. 


\section{Drafts}

Any member of the ESOSC is allowed to submit drafts for approval.
Any non-member of the ESOSC is allowed to request that the ESOSC should work
on a new standard and may submit suggestions or proposals for standards to
the ESOSC.  

\subsection{Proposals}

Proposals have no assigned EUIN. A proposal does not differ greatly from a draft.
A proposal is a draft from a non-member that has not yet received official draft
status. With receiving official draft status a proposal also receives an EUIN. \\


To achieve official draft status one member of the ESOSC must agree and assign
it a valid EUIN from their Assigned Number Block (ANB). 


\section{Revisions}

A new revision of an approved standard is put into the draft state and
requires the usual approval process taking place. The ESOSC is allowed to correct grammar,
orthographical or similar mistakes without requiring a new revision. A new revision
is created by creating a new document with the version number of the latest 
version incremented by one. The year and AN do not change. 


\section{New members}

For a non-member to become a member the candidate has to pass a vote. To pass the
vote 2/3 (rounded to the nearest integer rounding upwards at .5) of all existing 
members must vote to accept a candidate as a new member. A non-member must of
course agree to become a member. 

\subsection{Half members}

Half members can submit drafts and receive an Assigned Number Block, however,
half members can not vote and can not approve drafts. 

\subsection{Leaving}

Members or half members may leave the ESOSC at any point. It is recommended
to pay each remaining member (or half member) a beverage when doing so but this
is not mandatory. However, since most members don't actually know each other 
this is very unlikely to happen. 

It's also recommended to nominate a replacement. 


\section{Approval}

Approval of a standard does not mean the ESOSC recommends using the standard or
endorses it. Approval merely serves as a form of 'Quality Control' to ensure
that the standard is detailed enough for others to implement. THE ESOSC DOES
NOT VERIFY ANYTHING BEYOND THAT. 


\section{Settling disputes}

Should there be disputes about ESOSC related matters that are not well-defined
within this document such disputes must be settled democratically (by voting) requiring
a 2/3 (rounded to the nearest integer rounding upwards at .5) majority.


\section{This document}

This document governs the rules of ESOSC and can not be overridden by another
document. Changes to this standard (new revisions) requires approval from
a 2/3 (rounded to the nearest integer rounding upwards at .5) majority of members.


\section{Voting}

If voting takes place all members must be notified. If a member can't be reached for 
more than three weeks the member is presumed to abstain from voting (and thus does
not count for the calculation of the amount of votes required to reach majority). 


\section{Legal}

Any standards approved by the ESOSC are free to use and must be free to use. 
If a contributor holds a patent to parts of a standard they contributed to they
grant everybody an unrevocable right to use the patent free of charge without
limitations. However, the ESOSC can not guarantee that standards or part of 
standards are not patented by third parties. \\

STANDARDS ARE PROVIDED BY THE COPYRIGHT HOLDERS AND CONTRIBUTORS "AS IS" AND
ANY EXPRESS OR IMPLIED WARRANTIES, INCLUDING, BUT NOT LIMITED TO, THE IMPLIED
WARRANTIES OF MERCHANTABILITY AND FITNESS FOR A PARTICULAR PURPOSE ARE
DISCLAIMED. IN NO EVENT SHALL THE COPYRIGHT OWNER OR CONTRIBUTORS BE LIABLE FOR
ANY DIRECT, INDIRECT, INCIDENTAL, SPECIAL, EXEMPLARY, OR CONSEQUENTIAL DAMAGES
(INCLUDING, BUT NOT LIMITED TO, PROCUREMENT OF SUBSTITUTE GOODS OR SERVICES;
LOSS OF USE, DATA, OR PROFITS; OR BUSINESS INTERRUPTION) HOWEVER CAUSED AND
ON ANY THEORY OF LIABILITY, WHETHER IN CONTRACT, STRICT LIABILITY, OR TORT
(INCLUDING NEGLIGENCE OR OTHERWISE) ARISING IN ANY WAY OUT OF THE USE OF A
STANDARD, EVEN IF ADVISED OF THE POSSIBILITY OF SUCH DAMAGE.

\end{document}
